
Page 55
A large leading edge radius helps air stay attached at higher angles of attack, giving a higher stall angle and more lift for takeoff or landing. 
On the other hand, an overly fat leading edge gives more drag. There needs to be a compromise.

Page 56
Camber gives more lift at zero angle of attack and increases maximum lift of an airfoil, but it also increases drag and pitching moments. 
Tailless or flying-wing aircrafts need to use an "S"-shaped airfoil in order to gain natural stability. 

Page 56
Many airfoil analysis were separated into $2$ parts, the mean camber line and the thickness distribution.
The mean camber line will influence things like lift, induced drag, airfoil pitching moments
The thickness distribution will influence profile drag (separation drag).

Page 60
The aerodynamic center of the aircraft is typically located at the quarter chord.
At supersonic speeds however, the aerodynamic center moves backwards towards the trailing edge, it can move up to 0.4 of the airfoil chord in supersonic flight.
Center of pressure also moves backwards in supersonic flight.

Page 63
Modern airfoil design is based on an inverse problem.
Airfoils are designed such that the pressure differentials between the top and bottom surfaces reach maximum value without separation as quickly as possible. 
Towards the rear of the airfoil, various pressure recovery schemes are employed to prevent separation at the trailing edge.

Page 63
References on modern airfoils:
Wortmann, Eppler, Liebeck, Roncz

Page 67
Round Leading Edges and greater thickness over chord allows nice stall characteristics.
Stall starts fom trailing edge and slowly progresses to leading edge.
The loss of lift is gradual. Pitching moment only varies by a small amount.
%Seperator
Thinner airfoils are nasty. 
They stall from the leading edge and then stall immediately at some higher angle of attack
%Seperator
Very thin airfoils are nicer but not so good. 
They stall from the leading edge, and the separation bubble expands slowly to the trailing edge. 
Massive variation in pitch.

Page 77
Increasing the aspect ratio will reduce stall angle of attack.

Page 83
A taper ratio of 0.45 almost completely follows the ideal elliptical lift distribution.
When taking into accoun the weight due to taper, perhaps a taper ratio of 0.4 is ideal.

Page 88
Too much effective dihedral makes the aircraft have a dutch roll effect, which will need bigger horizontal stabilizers to negate, or anhedral to decrease roll stability.

Page 98 has stuff about tail arrangement
