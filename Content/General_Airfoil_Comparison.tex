%Seperator
%Seperator
%Seperator
%Seperator
%Seperator
\section{General Airfoil Comparisons}
\begin{comment}
\end{comment}

%Seperator
%Seperator
%Seperator
%Seperator
\subsection{"Fat" Airfoils}
\begin{comment}
\end{comment}
A "fat" airfoil is going to be defined as an airfoil with a large radius for its leading edge, is generally thick, and has a large amount of camber.

%Seperator
%Seperator
%Seperator
\subsubsection{Advantages}
\begin{comment}
\end{comment}

So the large camber of the airfoil allows it to produce alot of lift at zero angle of attack.
The large radius at its leading edge allows the airflow to remain attached, which increases the stall angle of the airfoil, allowing for a higher maximum lift coefficient.
The correct thickness distribution is helpful in preventing flow separation.
A thick airfoil has good stall characteristics. Stall will occur first in the trailing edge of the airfoil and progressively move to the leading edge. The loss of lift is gradual and pitching moments do not change by too much

%Seperator
%Seperator
%Seperator
\subsubsection{Disadvantages}
\begin{comment}
\end{comment}
It has alot of drag. The induced drag is higher because of its lifting capabilities, but its viscous drag is also higher because it has more surface area due to its camber and thickness. The pitching moments about the aerodynamic center is also great due to the camber of the airfoil.

%Seperator
%Seperator
%Seperator
%Seperator
\subsection{"Thin" Airfoils}
\begin{comment}
\end{comment}
A "thin" airfoil is going to be defined as an airfoil with a lower radius on its leading edge, is generally thin, and has a lower amount of camber.

%Seperator
%Seperator
%Seperator
\subsubsection{Advantages}
\begin{comment}
\end{comment}
These airfoils can suppress turbulence, maintaining laminar flow and getting very good lift to drag ratios. 
These airfoils typically produce a low amount of lift and also produce very mild pitching moments.

%Seperator
%Seperator
%Seperator
\subsubsection{Disadvantages}
\begin{comment}
\end{comment}
They cannot produce as much lift.
If they are taken out of their laminar bucket, drag will increase and lift will fall quickly, this is because laminar flows are more prone to separation.
They also exhibit terrible stall characteristics. Separation occurs complete and instantly, loss of lift is instant, and bad pitching moments will instantly appear during stall.

